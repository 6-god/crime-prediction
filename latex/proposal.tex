\documentclass[10pt]{article}
\usepackage{hyperref}
\usepackage{graphicx}
\usepackage{enumerate,fullpage,amsmath,amssymb}
\usepackage{hyperref}

\hypersetup{%
  colorlinks=true,% hyperlinks will be coloured
}

\title{CS 281 Final Project Proposal: Crime Predictions}
\author{\href{mailto:alexwang@college.harvard.edu}{Alex Wang} and \href{mailto:luisperez@college.harvard.edu}{Luis A. Perez}, Harvard University}
\date{\today}

\begin{document}

 \begin{center}
  \framebox{
    \vbox{
    \hbox to 6.5in { {CS 281: Advanced Machine Learning}
    \hfill Final Project Proposal}
    \vspace{4mm}
    \hbox to 6.5in { {\large \hfill Localized Crime Predictions \hfill} }
    \vspace{2mm}
    \hbox to 6.5in { {\it Alex Wang (alexwang@college.harvard.edu) \hfill Luis Antonio Perez (luisperez@college.harvard.edu)} }
    }
  }
  \end{center}

\section{Problem Statement}

    We are interested in applying machine learning methods to datasets regarding crime (crime statistics in particular cities) and possible related factors (such as tweet data, income, etc.). Specifically, we are interested in investigating if it is possible to predict criminal events for a specific time and place in the future (for example, assigning a risk level for a shooting within the next week to different neighborhoods). \\
    
	There is an abundance of data to incorporate into the model. Most notably, numerous cities, such as Chicago \footnote{\href{https://data.cityofchicago.org/}{City of Chicago Crime Dataset}. For the full link, see https://data.cityofchicago.org/} and New York \footnote{\href{https://nycopendata.socrata.com/Public-Safety/Historical-New-York-City-Crime-Data/hqhv-9zeg}{New York Crime Data}. For the full link, see https://nycopendata.socrata.com/Public-Safety/Historical-New-York-City-Crime-Data/hqhv-9zeg}, have large datasets detailing crimes over the past decade. This data is broken down into type of crime (e.g. theft, murder, narcotics, etc.), time of day and year, geographic location, and more. Furthermore, cities have released a myriad of other datasets that may possibly relate: median income, location of police stations, etc \footnote{These data sets appear to all be released through the same platform: \href{OpenData}{https://www.data.gov/open-gov/}}. Combined with possible other outside  sources, such as geo-located Tweets \footnote{\href{https://data.sfgov.org/Public-Safety/Map-Crime-Incidents-from-1-Jan-2003/gxxq-x39z}{San Francisco Crime Mapping} - Link:https://data.sfgov.org/Public-Safety/Map-Crime-Incidents-from-1-Jan-2003/gxxq-x39z}, newspaper headlines, etc) there is a copious amount of data to engineer features from and incorporate. This problem area is interesting because of the dynamics that drive crime and the level of granularity needed. \\
	
	Furthermore, this problem area is useful and novel. Being able to predict where crime will be most prevalent is of obvious benefit to a city and its citizens. The police force could allocate more resources to the predicted high-crime area in order to preempt criminal activity, and citizens would be safer by knowing what parts of a city they should avoid. However, in order to create actionable insights, the model created needs to be able to predict crime within a small time frame, for example around a day or week, and a small geographic region, e.g. a block or a neighborhood. \\
	
	Having described the available data and the possible difficulties, we can now refine the questions to investigate in this project. First and foremost, we are interested in seeing if we can predict the criminal incidents, perhaps for a specific type of crime, for a small time frame and geographic region. Second, we are interested in learning which features have the most predictive power with respect to crime. Having an understanding of driving factors, cities can better work to mitigate the risk factors for crime.

\section{Approach}
	As a problem-driven project, a significant portion of this project will include implementing various regressions algorithms. The first challenge in this problem will be feature extraction. PCA and other methods to analyze correlation between features we extract, as well as simple data exploration methodologies (such as analyzing trends over-time, etc.) will be helpful for feature prediction. \\
	
	For ML models, we will begin with simple linear regression (Bayesian) on the features we ultimately extract. We expect numerous features, so we will need to enforce sparsity via regularization or some form of dimensionality reduction. Ultimately we would like to try to implement a neural network, specifically a recurrent neural network, for practice and to see if they lead to improved performance. Our understanding is that RNNs are good for capturing patterns in time series data, which is how this data is structured. In that case we would implement the neural net in Torch. Beyond that, we will need to explore various other regression models to determine what is appropriate for the data and context. If time allows, we would also like to explore other models typically good for time series data, such as Hidden Markov Models or Gaussian Process, though we're somewhat unfamiliar with both. \\
	
	In the end, we will likely need to develop a more sophisticated model that more fully captures the dynamics of crime. For example, some research has been done into the mutually excitatory nature of crime, e.g. crime in one area causes surrounding areas to be more susceptible to crime. These types of modeling enhancements may lead to better predictive power. We expect to spend a good amount of time tweaking different model ideas, and measuring their performance.

\section{Evaluation}
	The current standard in crime forecasting is hotspot mapping, the process of mapping out occurrences of crime to identify high crime areas. The thought, then, is that placing more officers or other preventative resources in that areas with historically high rates of crime will more significantly reduce crime. Though simple, in some specific cases, this method had 90\% accuracy when using the previous year's data to forecast the next year, according to one study. Combined with the fact that hotspot mapping is easy to do with modern software and easy to understand, it is certainly the baseline to beat. \\
	
	The metric we plan on using is root-mean-square error between the predicted value and the true value for a given area. Since cities' crime data goes back over a decade, we plan on withholding a few years' data as a test set. The baseline machine learning algorithm will be simple Bayesian regression on our selected feature space. We're still in the process of determining the evaluation plans for our more ambitious models, such as a RNN or a HMM.\\
	
	There has been some work done already on this topic \footnote{\href{http://ieeexplore.ieee.org.ezp-prod1.hul.harvard.edu/stamp/stamp.jsp?tp=&arnumber=5580971}{A Novel Serial Crime Prediction Model Based on Bayesian Learning Theory }} \footnote{\href{http://delivery.acm.org.ezp-prod1.hul.harvard.edu/10.1145/2670000/2663254/p427-bogomolov.pdf?ip=128.103.149.52&id=2663254&acc=ACTIVE\%20SERVICE&key=C82FBC3DCC335AD2\%2E4D4702B0C3E38B35\%2E4D4702B0C3E38B35\%2E4D4702B0C3E38B35&CFID=553305163&CFTOKEN=20331583&__acm__=1444871399_feb25f6bb65f577ea58bb1f29f2f176a}{Once Upon a Crime: Towards Crime Prediction from
Demographics and Mobile Data}}, but not much. We were able to find one review paper that detailed various regression techniques employed already, including neural networks, but many of these were dated before the release of these large datasets and before the development of different flavors of neural nets. One recent paper\footnote{\href{http://link.springer.com/chapter/10.1007/978-3-642-29047-3_28}{Automatic Crime Prediction Using Events Extracted from Twitter Posts}} we looked at incorporating Tweets into a model and found improvements, though they also did not incorporate all the data available at the time, suggesting that there is ample room for exploration in this domain.


\section{Collaboration Plan}
The work primarily falls into two divisions: feature engineering and machine learning. We both would like to do both aspects, so our plan is to each code up different regression models and each engineer features from different feature sets. 

\begin{itemize}
\item[Luis] Feature extraction from data sets, with particular emphasis on newspaper headline extraction and data set merging (merge crime statistics with income, etc.). Models to test include Gausiann Process and Hidden Markov Models (for clustering).
\item[Alex] Feature extraction, in particular time series-like features, with focus on specific types of crime. Likely to explore Bayesian linear regression, as well as RNN.
\end{itemize}
The code will be hosted publicly on gitHub (open source). The repo can be found here: 
\end{document}

